\usepackage[utf8]{inputenc}
\usepackage[english]{babel}
\usepackage{amssymb,amsmath}
\usepackage{mathpartir}
\usepackage{latexsym}
\usepackage{stmaryrd}
\usepackage{semantic}
\usepackage{graphicx}
\usepackage{eurosym}
\usepackage{hyperref}
\usepackage{array}
\usepackage{multirow}
\usepackage{mathtools}
\usepackage{centernot}
\usepackage{tikz}
\usepackage{pgffor}
\usepackage{enumerate}


\newcommand{\currbreakcontract}[2]{\text{break}_0(#1,#2)}
\newcommand{\currdangerofbreakingcontract}[2]{\text{endanger}_0(#1,#2)}
\newcommand{\breakcontract}[2]{\text{break}(#1,#2)}
\newcommand{\dangerofbreakingcontract}[2]{\text{endanger}(#1,#2)}

%\theoremstyle{definition}
\newtheorem{definition}{Definition}[section]
\newtheorem{theorem}{Theorem}[section]
\newtheorem{example}{Example}[section]
\newtheorem{proposition}{Proposition}[section]
\newtheorem{corollary}{Corollary}[section]
%\newenvironment{proof}{%
%  \upshape\begin{list}{}
%    \item \emph{Proof.}
%}%
%{\end{list}}

%\makeatletter
\usetikzlibrary{calc,automata,shapes.multipart, positioning, shapes.arrows,positioning,calc,trees}
\tikzset{
  state new/.style={circle, draw, align=center},
%  state/.style={circle mio, draw, text width=0.3cm, align=center},
  state/.style={circle, draw, align=center},
  line/.style = {->,>=stealth,line width=0.4pt, shorten <=1mm, shorten >=1mm},
  line over/.style = {preaction={draw=white, -,line width=3pt}, line},
  shortcut/.style = {line, dashed},
  every edge/.append style={line},
  conode/.style={rectangle split, 
    rectangle split horizontal, 
    rectangle split parts=3,draw},
  cotree/.style={fill,
    edge from parent path={(\tikzparentnode\tikzparentanchor) 
      -- +(0em,-0.5\tikzleveldistance) 
      node[pos=1,circle,inner sep=0pt,minimum
      size=4pt,fill,fill=white, draw]{}
      [->, >=stealth, rounded corners=2pt] -| 
      (\tikzchildnode\tikzchildanchor)}
  },
  coagent/.style={inner sep=1pt, anchor=south west},
  coname/.style={inner sep=1pt, anchor=north east}
}

\pgfdeclareshape{circle mio}
{
  %
  % Node parts
  %
  \nodeparts{text,lower}

  %
  % Anchors
  %
  \savedanchor\centerpoint{%
    \pgf@x=.5\wd\pgfnodeparttextbox%
    \pgfmathsetlength{\pgf@y}{\pgfkeysvalueof{/pgf/inner ysep}}%
    \pgf@y=-\pgf@y%
    \advance\pgf@y by-\dp\pgfnodeparttextbox%
    \advance\pgf@y by-.5\pgflinewidth%
  }%
  \savedanchor\loweranchor{%
    \pgf@x=-.5\wd\pgfnodepartlowerbox%
    \advance\pgf@x by.5\wd\pgfnodeparttextbox%
    \pgfmathsetlength{\pgf@y}{\pgfkeysvalueof{/pgf/inner ysep}}%
    \pgf@y=-2\pgf@y%
    \advance\pgf@y by-\ht\pgfnodepartlowerbox%
    \advance\pgf@y by-.5\pgflinewidth%
    \advance\pgf@y by-\dp\pgfnodeparttextbox%
    \advance\pgf@y by-.5\pgflinewidth%
  }
    
  \saveddimen\radius{%
    % 
    % Caculate ``height radius''
    %
    %\pgf@ya=.5\ht\pgfnodeparttextbox%
%    \advance\pgf@ya by.5\dp\pgfnodeparttextbox%
%    \advance\pgf@ya by.5\ht\pgfnodepartlowerbox%
%    \advance\pgf@ya by.5\dp\pgfnodepartlowerbox%
%    \advance\pgf@ya by.5\pgflinewidth%
                %
                % MW: Suggested correction for above calculation:       Use the tallest box * 2.
                %
                \pgf@ya=.5\ht\pgfnodeparttextbox%
                \advance\pgf@ya by.5\dp\pgfnodeparttextbox%
                \pgf@yb=.5\ht\pgfnodepartlowerbox%
                \advance\pgf@yb by.5\dp\pgfnodepartlowerbox%
                \ifdim\pgf@ya>\pgf@yb\relax%
                        \pgf@ya2.0\pgf@ya\relax%
                \else%
                        \pgf@ya2.0\pgf@yb\relax%
                \fi%
                \advance\pgf@ya by.5\pgflinewidth%
    \pgfmathsetlength\pgf@yb{\pgfkeysvalueof{/pgf/inner ysep}}%
    \advance\pgf@ya by2\pgf@yb%
    % 
    % Caculate ``width radius''
    % 
    \pgf@xa=.5\wd\pgfnodeparttextbox%
    \ifdim\pgf@xa<.5\wd\pgfnodepartlowerbox%
      \pgf@xa=.5\wd\pgfnodepartlowerbox%
    \fi%
    \pgfmathsetlength\pgf@xb{\pgfkeysvalueof{/pgf/inner xsep}}%
    \advance\pgf@xa by\pgf@xb%
    % 
    % Calculate length of radius vector:
    % 
    \pgf@process{\pgfpointnormalised{\pgfqpoint{\pgf@xa}{\pgf@ya}}}%
    \ifdim\pgf@x>\pgf@y%
        \c@pgf@counta=\pgf@x%
        \ifnum\c@pgf@counta=0\relax%
        \else%
          \divide\c@pgf@counta by 255\relax%
          \pgf@xa=16\pgf@xa\relax%
          \divide\pgf@xa by\c@pgf@counta%
          \pgf@xa=16\pgf@xa\relax%
        \fi%
      \else%
        \c@pgf@counta=\pgf@y%
        \ifnum\c@pgf@counta=0\relax%
        \else%
          \divide\c@pgf@counta by 255\relax%
          \pgf@ya=16\pgf@ya\relax%
          \divide\pgf@ya by\c@pgf@counta%
          \pgf@xa=16\pgf@ya\relax%
        \fi%
    \fi%
    \pgf@x=\pgf@xa%
    % 
    % If necessary, adjust radius so that the size requirements are
    % met: 
    % 
    \pgfmathsetlength{\pgf@xb}{\pgfkeysvalueof{/pgf/minimum width}}%  
    \pgfmathsetlength{\pgf@yb}{\pgfkeysvalueof{/pgf/minimum height}}%  
    \ifdim\pgf@x<.5\pgf@xb%
        \pgf@x=.5\pgf@xb%
    \fi%
    \ifdim\pgf@x<.5\pgf@yb%
        \pgf@x=.5\pgf@yb%
    \fi%
    % 
    % Now, add larger of outer sepearations.
    % 
    \pgfmathsetlength{\pgf@xb}{\pgfkeysvalueof{/pgf/outer xsep}}%  
    \pgfmathsetlength{\pgf@yb}{\pgfkeysvalueof{/pgf/outer ysep}}%  
    \ifdim\pgf@xb<\pgf@yb%
      \advance\pgf@x by\pgf@yb%
    \else%
      \advance\pgf@x by\pgf@xb%
    \fi%
  }

  %
  % Anchors
  % 
  \inheritanchorborder[from=circle]
  \inheritanchor[from=circle]{north}
  \inheritanchor[from=circle]{north west}
  \inheritanchor[from=circle]{north east}
  \inheritanchor[from=circle]{center}
  \inheritanchor[from=circle]{west}
  \inheritanchor[from=circle]{east}
  \inheritanchor[from=circle]{mid}
  \inheritanchor[from=circle]{mid west}
  \inheritanchor[from=circle]{mid east}
  \inheritanchor[from=circle]{base}
  \inheritanchor[from=circle]{base west}
  \inheritanchor[from=circle]{base east}
  \inheritanchor[from=circle]{south}
  \inheritanchor[from=circle]{south west}
  \inheritanchor[from=circle]{south east}
  \anchor{lower}{\loweranchor}

  %
  % Background path
  %
  \inheritbackgroundpath[from=circle]
  \beforebackgroundpath{
    \pgfutil@tempdima=\radius%
    \pgfmathsetlength{\pgf@xb}{\pgfkeysvalueof{/pgf/outer xsep}}%  
    \pgfmathsetlength{\pgf@yb}{\pgfkeysvalueof{/pgf/outer ysep}}%  
    \ifdim\pgf@xb<\pgf@yb%
      \advance\pgfutil@tempdima by-\pgf@yb%
    \else%
      \advance\pgfutil@tempdima by-\pgf@xb%
    \fi%
    \advance\pgfutil@tempdima by-.5\pgflinewidth%  
    \pgfsetshortenstart{0pt}%
    \pgfsetshortenend{0pt}%
    \pgfsetarrows{-}%  
    \pgfpathmoveto{\pgfpointadd{\centerpoint}{\pgfqpoint{-0.8\pgfutil@tempdima}{0pt}}}%
    \pgfpathlineto{\pgfpointadd{\centerpoint}{\pgfqpoint{0.8\pgfutil@tempdima}{0pt}}}%
    \pgfusepath{stroke}%
  }
}


\newcommand{\conode}[3]{#1\nodepart{two}#2\nodepart{three}#3}
\newcommand{\stcond}[2]{#1\nodepart{lower}#2}


\newcommand{\true}{\ensuremath{\mathit{tt}}}
\newcommand{\false}{\ensuremath{\mathit{ff}}}
\newcommand{\tracelength}[1]{\##1}


\newcommand{\augmented}[1]{#1_{\mbox{\footnotesize cond\normalsize}}}
\newcommand{\concat}{+\!\!+}
\newcommand{\st}{\;\cdot\;}
\newcommand{\df}{\stackrel{\mbox{\footnotesize df\normalsize}}{=}}
\newcommand{\subsumed}{\stackrel{\mbox{\tiny$\Leftarrow$\normalsize}}{=}}
\newcommand{\normalform}{\mbox{nf}}
\newcommand{\QQ}{{\mathcal Q}}
\newcommand{\prA}{\mathcal{A}}
\newcommand{\refinedby}[2]{\sqsubseteq_{\mbox{\footnotesize\emph{#1}}}^#2}

\newcommand{\refinedbyMustV}{\refinedby{must}{v}}
\newcommand{\refinedbyMayV}{\refinedby{may}{v}}

\newcommand{\refinedbyMustD}{\refinedby{must}{d}}
\newcommand{\refinedbyMayD}{\refinedby{may}{d}}

\newcommand{\mustviolatetraces}{\mbox{traces}_{\mbox{{\footnotesize must}}}^v}
\newcommand{\mayviolatetraces}{\mbox{traces}_{\mbox{{\footnotesize may}}}^v}

\newcommand{\mustdangertraces}{\mbox{traces}_{\mbox{{\footnotesize must}}}^d}
\newcommand{\maydangertraces}{\mbox{traces}_{\mbox{{\footnotesize may}}}^d}


\def\squareforqed{\hbox{\rlap{$\sqcap$}$\sqcup$}}
\def\qed{%
     \ifhmode%
        \unskip\nobreak\hfil%
        \penalty50\hskip1em\null\nobreak\hfil\squareforqed%
        \parfillskip=0pt\finalhyphendemerits=0\endgraf%
     \else%
       % \nopagebreak[4]%
       % \leavevmode\nobreak\hfill\squareforqed%
       \kern-1em\samepage\leavevmode\hfill\squareforqed%
   \fi}

% \newtheorem{mytheorem}{Theorem}
% \newtheorem{mydefinition}{Definition}
% \newtheorem{myexample}{Example}
% \newtheorem{mylemma}{Lemma}
% \newtheorem{myproposition}{Proposition}
% \newtheorem{mycorollary}{Corollary}
%\newenvironment{proof}{\par\textit{Proof.}\ }{}

% \makeatletter
% \let\old@proposition\proposition
% \let\old@endproposition\endproposition
% \renewenvironment{proposition}{\old@proposition\upshape}{\qed\old@endproposition}
% \let\old@theorem\theorem
% \let\old@endtheorem\endtheorem
% \renewenvironment{theorem}{\old@theorem\upshape}{\qed\old@endtheorem}
% \let\old@definition\definition

% \let\old@enddefinition\enddefinition
% \renewenvironment{definition}{\old@definition\upshape}{\qed\old@enddefinition}
% \let\old@lemma\lemma
% \let\old@endlemma\endlemma
% \renewenvironment{lemma}{\old@lemma\upshape}{\qed\old@endlemma}
% \let\old@corollary\corollary
% \let\old@endcorollary\endcorollary
% \renewenvironment{corollary}{\old@corollary\upshape}{\qed\old@endcorollary}
% \newenvironment{proofappendix}[2]{
%   \noindent\textit{\textbf{Proof of #1~\ref{#2}}}\\
% }{\qed}
% \let\old@example\example
% \let\old@endexample\endexample
% \renewenvironment{example}{\old@example}{\old@endexample\qed}

%\newenvironment{theorem}{\begin{mytheorem}\upshape}{\qed\end{mytheorem}}
%\newenvironment{definition}{\begin{mydefinition}\upshape}{\qed\end{mydefinition}}
%\newenvironment{lemma}{\begin{mylemma}\upshape}{\qed\end{mylemma}}
%\newenvironment{proposition}{\begin{myproposition}\upshape}{\qed\end{myproposition}}
%\newenvironment{corollary}{\begin{mycorollary}\upshape}{\qed\end{mycorollary}}
%\newenvironment{example}{\begin{myexample}\upshape}{\qed\end{myexample}}
\makeatother


\DeclareMathSymbol\gusano{\mathrel}{lasy}{"3A}

\newbox\arriba
\newbox\abajo
\newbox\CaracterInterno
\newbox\CaracterDerecha
\newdimen\anchura
\def\MacrosTranGeneral#1#2#3#4#5#6{%
  \setbox\CaracterInterno=\hbox{\mathsurround=0pt$\mathord#4$}
  \setbox\CaracterDerecha=\hbox{\mathsurround=0pt$\mathord#3$}
  \setbox\arriba=\hbox{$#1#2$}
  \setbox\abajo=\hbox{\mathsurround=0pt%
                      \anchura=\wd\arriba%
                      \advance \anchura by 0.5em%
                      \divide \anchura by \wd\CaracterInterno%
                      \multiply \anchura by \wd\CaracterInterno%
                      \copy\CaracterInterno\kern\SeparacionInternaFlecha
                      \hbox to \anchura{%
                          $\cleaders%
                            \hbox{\kern\SeparacionInternaFlecha\copy\CaracterInterno}
                            \hfill$}%
                      \kern\SeparacionExternaFlecha\copy\CaracterDerecha}
  \mathrel{{\buildrel\vbox{\copy\arriba \kern\SeparacionFlechaArriba} %
    \over{\copy\abajo^{#6}}}_{#5}}
  }
\def\MacrosTranGeneralProp#1#2#3#4#5{\mathchoice%
  {\MacrosTranGeneral{\scriptstyle}{#1}{#2}{#3}{#4}{#5}}
  {\MacrosTranGeneral{\scriptstyle}{#1}{#2}{#3}{#4}{#5}}
  {\MacrosTranGeneral{\scriptscriptstyle}{#1}{#2}{#3}{#4}{#5}}
  {\MacrosTranGeneral{\scriptscriptstyle}{#1}{#2}{#3}{#4}{#5}}}

\def\MacrosTran#1{%
  \def\SeparacionInternaFlecha{-0.3em}%
  \def\SeparacionExternaFlecha{-0.5em}%
  \def\SeparacionFlechaArriba{-3pt}%
  \MacrosTranGeneralProp{#1}{{\rightarrow}}{{-}}{}{}}
\def\MacrosNoTran#1{%
  \def\SeparacionInternaFlecha{-0.3em}
  \def\SeparacionExternaFlecha{-0.5em}
  \def\SeparacionFlechaArriba{-3pt}
  \MacrosTranGeneralProp{#1\kern 0.5em}{{\not\rightarrow}}{{-}}{}{}}
\def\MacrosVTran#1{%
  \def\SeparacionInternaFlecha{-0.2em}
  \def\SeparacionExternaFlecha{-0.5em}
  \def\SeparacionFlechaArriba{0pt}
  \MacrosTranGeneralProp{#1}{{\Rightarrow}}{{=}}{}{}}
\def\MacrosNoVTran#1{%
  \def\SeparacionInternaFlecha{-0.2em}
  \def\SeparacionExternaFlecha{-0.5em}
  \def\SeparacionFlechaArriba{-3pt}
  \MacrosNoTranGeneralProp{#1}{{\not\Rightarrow}}{{=}}{}{}}
\def\MacrosTiTran#1{%
  \def\SeparacionInternaFlecha{-0.0pt}
        \def\SeparacionExternaFlecha{-0.0pt}
        \def\SeparacionFlechaArriba{0pt}
  \kern 1pt\MacrosTranGeneralProp{#1}{{\leadsto}}{{\gusano}}{}{}}
\def\MacrosTiTranRe#1{%
  \def\SeparacionInternaFlecha{-0pt}
        \def\SeparacionExternaFlecha{-0.0pt}
        \def\SeparacionFlechaArriba{-3pt}
  \kern 1pt \MacrosTranGeneralProp{#1}{{\leadsto}}{{\gusano}}{}{*}}
\def\MacrosNoTiTranRe#1{%
  \def\SeparacionInternaFlecha{-0pt}
        \def\SeparacionExternaFlecha{-0.0pt}
        \def\SeparacionFlechaArriba{-3pt}
  \kern 1pt \MacrosTranGeneralProp{#1}{{\not\leadsto}}{{\gusano}}{}{*}}
\def\MacrosNoTiTran#1{%
  \def\SeparacionInternaFlecha{-0pt}
        \def\SeparacionExternaFlecha{-0.0pt}
        \def\SeparacionFlechaArriba{-3pt}
  \kern 1pt \MacrosTranGeneralProp{#1}{{\not\leadsto}}{{\gusano}}{}{}}



%\def\tran#1{\ensuremath\mathbin{\MacrosTran{#1}}}
\def\vtran#1{\ensuremath\mathbin{\MacrosVTran{#1}}}
\def\notran#1{\ensuremath\mathbin{\MacrosNoTran{#1}}}
\def\novtran#1{\ensuremath\mathbin{\MacrosNoVTran{#1}}}
\newcommand{\tran}[1]{\MacrosTran{#1}}
\newcommand{\transys}{\ensuremath\mathbin{\MacrosTiTran{}}}
\newcommand{\transysre}{\MacrosTiTranRe{}}
\newcommand{\notranti}[1]{\MacrosNoTiTran{#1}}
\newcommand{\notrantire}[1]{\MacrosNoTiTranRe{#1}}


%% For iocos stuff.
\newcommand{\mIUT}{\ensuremath{\mathrm{IUT}_m}}
\newcommand{\Onodelta}{\ensuremath{O_{\not\delta}}}

\newcommand{\tranp}[2]{\MacrosTran{#1}_{\!\!#2}\;}
\newcommand{\tranap}[2]{\MacrosVTran{#1}_{\!\!#2}\;}
\newcommand{\tranb}[2]{\stackrel{#1}{\Longrightarrow}_{#2}}
\newcommand{\transi}[2]{\MacrosTran{#1}_{\!\!#2}\;}
\newcommand{\iocos}{\ensuremath{\mathbin{\mathsf{ioco\underline{s} }}}}
\newcommand{\giocos}{\ensuremath{\mathbin{\mathsf{g\mbox{-}ioco\underline{s} }}}}
\newcommand{\niocos}{\ensuremath{\mathbin{\kern 1em\hbox to 0pt{/\hss}\kern-1em \mathsf{ioco\underline{s}}}}}
%\newcommand{\iocoqs}{\mathbin{\mathbf{iocoqs}}}
\newcommand{\ioco}{\ensuremath{\mathbin{\mathsf{ioco}}}}
\newcommand{\nioco}{\ensuremath{\mathbin{\kern 1em\hbox to 0pt{/\hss}\kern-1em \mathsf{ioco}}}}
\newcommand{\IOTS}{\ensuremath{\mathit{IOTS}}}
\newcommand{\LTS}{\ensuremath{\mathit{LTS}}}
\newcommand{\LTSs}{\ensuremath{\mathit{LTSs}}}
\newcommand{\Lts}{\ensuremath{{\cal L}}}
\newcommand{\after}{\mathbin{\mathsf{after}}}
\newcommand{\partes}{\ensuremath{{\cal P}}}
% \newcommand{\success}{\hbox{\blacksmiley}}
% \newcommand{\fail}{\hbox{\frownie}}
% \newcommand{\success}{\hbox{\Checkmark}}
% \newcommand{\fail}{\hbox{\XSolidBrush}}
% \newcommand{\success}{\checkmark}
% \newcommand{\fail}{\hbox{X}}
\newcommand{\success}{\hbox{\fontsize{8}{10}\selectfont\Checkmark}}
\newcommand{\fail}{\lower2pt\hbox{\fontsize{8}{10}\selectfont\XSolidBrush}}

\newcommand{\pass}{\mathbin{\mathsf{pass}}}
\newcommand{\nopass}{\mathbin{\kern1em\hbox to 0pt{/\hss}\kern-1em \mathsf{pass}}}
\renewcommand{\emptyset}{\varnothing}

%\renewcommand{\after}{\mathbin{~\mathsf{after}~}}
\newcommand{\traces}{\mathsf{traces}}
\newcommand{\Straces}{\mathit{Straces}}
\newcommand{\outs}{\mathsf{outs}}
\newcommand{\ins}{\mathsf{ins}}
\newcommand{\init}{\mathsf{init}}

%% General commands

\newcommand{\comen}[1]{}
%\newcommand{\nota}[1]{\textbf{#1}}


\newcommand{\tests}{\ensuremath{\mathit{{\cal T}}}}
\newcommand{\vtests}{\ensuremath{\mathit{{\cal T}}_v}}
\newcommand{\ltest}{\ensuremath{\mathbin{\sqsubseteq_T}}}
\newcommand{\noltest}{\ensuremath{\mathbin{\kern0.5em/\kern-1em \sqsubseteq_T}}}
\newcommand{\testfrom}[1]{{\cal T}(#1)}
\newcommand{\partini}{\Omega}
\newcommand{\relini}{\preccurlyeq}
\newcommand{\nat}{\mathbb{N}}
\newcommand{\calS}{\cal S}
\newcommand{\recup}{\vartriangleright}
\newcommand{\fin}{\;\mathsf{in}\;}
\newcommand{\vc}{\mathsf{vc}}
\newcommand{\ic}{\mathsf{ic}}
\newcommand{\paral}{\mathbin{\|}}
\newcommand{\choice}{\mathbin{\square}}
\newcommand{\ichoice}{\mathbin{\sqcap}}
\newcommand{\rec}{\mathsf{rec\;}}
%\newcommand{\dotminus}{\mathbin{\vbox{\hbox{.}\kern-0.8em\hbox{-}}}}
\newcommand{\dotminus}{\mathbin{\vbox{\hbox to 1em{\hfill$.$\hfill}\kern-0.8em\hbox to 1em{\hfill$-$\hfill}}}}
\newcommand{\act}{\mathsf{Act}}
\newcommand{\acttau}{\act_\tau}
\newcommand{\timedomain}{{\cal T}}
\newcommand{\agentset}{\ensuremath{{\cal AG}}}
\newcommand{\lsim}{\lesssim}
\newcommand{\cent}{c}
\newcommand{\pr}{\mathsf{P_r}}
\newcommand{\ob}{\mathsf{O_b}}
\newcommand{\fb}{\mathsf{F_b}}
\newcommand{\regla}[1]{\ensuremath{\textsf{#1}}}
\newcommand{\var}{\mathsf{Var}}
\newcommand{\wait}{\mathsf{Wait}}
\newcommand{\sync}[1]{\overline{#1}}
\makeatletter
\def\@inferenceFrontName[#1]{%
  \setbox3=\hbox{\textsf{\footnotesize #1}}%
  \ifdim \wd3 > \z@
    \unhbox3%
    \hskip\@@nSpace
  \fi
  \@inferenceMiddle
}

\def\sat{\vdash}
\def\violate#1{\vdash_{\mathit{v#1}}}
\def\violmay{\violate{may}}
\def\violmust{\violate{must}}

\newcommand{\oo}[2]{{\mathcal{O}_{#1}(#2)}}
\newcommand{\ou}[4]{{\mathcal{O}_{#1}(#2){\cal U}[#4,#3]}}
\newcommand{\ff}[2]{{\mathcal{F}_{#1}(#2)}}
\newcommand{\fu}[4]{{\mathcal{F}_{#1}(#2){\cal U}[#4,#3]}}
\newcommand{\pp}[2]{{\mathcal{P}_{#1}(#2)}}

\newcommand{\processes}[1]{\|_{i\in I} #1_i}
\newcommand{\Aprocesses}{{\mathcal A}}
\newcommand{\transition}[2]{\xrightarrow{#1,\;\{#2\}}}
\newcommand{\notransition}[2]{\centernot{\transition{#1}{#2}}}
\newcommand{\system}[2]{\langle #1,\;#2\rangle}

\newcommand{\name}[1]{\ensuremath{\mathbf{#1}}}
\newcommand{\rulenamed}[3]{\name{#1}\ \frac{#2}{#3}}

\let\oldphi\phi
\def\phi{\varphi}
%\def\genrec{\mathbin{\hbox to 0pt{$\cdot$\hss}\triangleright}}
\def\genrec{\ogreaterthan}
\def\dangerrec{\triangleright}
\def\violationrec{\blacktriangleright}
\def\violation{\mathsf{violation}}
\def\danger{\mathsf{danger}}

\def\calA{{\cal A}}
\def\calI{{\cal I}}
\def\calC{{\cal C}}
\def\calP{{\cal P}}
\newcommand\mor{\vee}
\newcommand\mand{\wedge}
\def\implies{\longrightarrow}
\makeatother

\newcommand{\dval}{\mathbin{\vdash_{\kern -0.3em d}}}
\newcommand{\vval}{\mathbin{\vdash_{\kern -0.3em v}}}
\newcommand{\vio}{\mathsf{vio}}
\newcommand{\dan}{\mathsf{dan}}
\newcommand{\cset}{\calC}
\newcommand{\pred}{\calP}
\newcommand{\tolerant}{\mathbin{\preccurlyeq}}
\newcommand{\cequiv}{\backsim}
\makeatother
\newcommand{\ci}{\ensuremath{\mathtt{checkin}}}
\newcommand{\landing}{\ensuremath{\mathtt{landing}}}
\newcommand{\boarding}{\ensuremath{\mathtt{boarding}}}

\newcommand{\pbp}{\ensuremath{\mathtt{PBP}}}
\newcommand{\gbch}{\ensuremath{\mathtt{GBCh}}}
\newcommand{\shp}{\ensuremath{\mathtt{ShP}}}
\newcommand{\weapon}{\ensuremath{\mathtt{weapon}}}
\newcommand{\board}{\ensuremath{\mathtt{board}}}
\newcommand{\liq}{\ensuremath{\mathtt{liq}}}
\newcommand{\hl}{\ensuremath{\mathtt{hl}}}
\newcommand{\hlhold}{\ensuremath{\mathtt{hlhold}}}
\newcommand{\dliq}{\ensuremath{\mathtt{dliq}}}
\newcommand{\cotext}{\fontsize{6}{7}\selectfont}
\newcommand{\coand}{\cotext \textsf{AND}}
\newcommand{\coseq}{\cotext \textsf{SEQ}}
\newcommand{\only}[1]{\mathsf{only(#1)}}
\newcommand{\cond}[3]{[#1](#2,#3)}
\newcommand{\until}[2]{#1\ {\cal U}[#2]}

%\newtheorem{corollary}{Corollary}

\newcommand{\calD}{{\mathcal D}}
\newcommand{\closure}[1]{\textit{closure}(#1)}
\newcommand{\fulltran}[1]{\tran{#1}\!_{\textit{full}}\;}

%%% Local Variables:
%%% mode: latex
%%% TeX-master: "main.tex"
%%% End:
